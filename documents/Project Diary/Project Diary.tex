\documentclass[12pt]{article}
\usepackage{color}
\usepackage{hyperref}
\pagestyle{plain}

\title{Programmierprojekt FS20\\Bachelor Chasers\\Diary}
\author{\\Adrian Prokopczyk\\Johannes Nussbaum\\Anna Diack\\Meipei Nghiem}
\date{}

\begin{document}

\begin{titlepage}

\maketitle
\thispagestyle{empty}
\setcounter{tocdepth}{2}

\end{titlepage}

\paragraph{Eintrag 1:}
20.02.2020 \\
Erster Termin der Einf\"uhrungsvorlesung. Sehr rasch nachdem wir unsere Gruppe gebildet hatten, einigten wir uns auf Black Jack als Basis für unser Spiel. Um es etwas spannender zu gestalten, m\"ochten wir Begriffe aus dem Studentenleben einbauen: Das Ziel ist es, 180 Punkte zu sammeln ("Bachelor"). Die Karten k\"onnten benannt werden nach Ereignissen im Studentenleben: Kaffee und Lernen gibt Pluspunkte; Partymachen und Schw\"anzen gibt Minuspunkte.\\
Wir erstellen eine WhatsApp-Gruppe zur Kommunikation untereinander.

\paragraph{Eintrag 2:}
26.02.2020\\
Vor der ersten \"Ubungsstunde trafen wir uns, um uns Gedanken zu machen \"uber unser Spiel. Wir beschliessen einige Veränderungen und Verfeinerungen, die Anna in einem Word-Dokument festh\"alt. Der Name unseres Spiels wird "Bachelor Chase" sein, und wir sind die "Chaser".\\
Als n\"achstes Treffen (neben den Vorlesungsstunden) wird Do, 5. M\"arz, 10 Uhr an der Spiegelgasse festgelegt.\\
Vorl\"aufig gelten folgende Zust\"andigkeiten:
\begin{itemize}
\item Anna bereitet die PowerPoint für den Meilenstein 1 vor.
\item Meipei macht das Mock-up unseres Spiels.
\item Johannes wird während der ganzen Projektdauer das Tagebuch betreuen.
\end{itemize}

\paragraph{Eintrag 3:}
28.02.2020\\
Meipei fügt unserem Repository ein gitignore-File hinzu und löscht die überflüssigen Dateien.\\
Adrian wird einen ersten Entwurf des Netzwerkprotokolls anfertigen und ins Repository stellen.

\end{document}