\documentclass[12pt]{article}
\usepackage{color}
\usepackage{hyperref}
\pagestyle{plain}

\title{Programmierprojekt FS20\\Bachelor Chasers\\Diary}
\author{\\Adrian Prokopczyk\\Johannes Nussbaum\\Anna Diack\\Meipei Nghiem}
\date{}

\begin{document}

\begin{titlepage}

\maketitle
\thispagestyle{empty}
\setcounter{tocdepth}{2}

\end{titlepage}

\paragraph{Eintrag 1:}
20.02.2020 \\
Erster Termin der Einf\"uhrungsvorlesung. Sehr rasch nachdem wir unsere Gruppe gebildet hatten, einigten wir uns auf Black Jack als Basis für unser Spiel. Um es etwas spannender zu gestalten, m\"ochten wir Begriffe aus dem Studentenleben einbauen: Das Ziel ist es, 180 Punkte zu sammeln ("Bachelor"). Die Karten k\"onnten benannt werden nach Ereignissen im Studentenleben: Kaffee und Lernen gibt Pluspunkte; Partymachen und Schw\"anzen gibt Minuspunkte.\\
Wir erstellen eine WhatsApp-Gruppe zur Kommunikation untereinander.

\paragraph{Eintrag 2:}
26.02.2020\\
Vor der ersten \"Ubungsstunde trafen wir uns, um uns Gedanken zu machen \"uber unser Spiel. Wir beschliessen einige Veränderungen und Verfeinerungen, die Anna in einem Word-Dokument festh\"alt. Der Name unseres Spiels wird "Bachelor Chase" sein, und wir sind die "Chaser".\\
Als n\"achstes Treffen (neben den Vorlesungsstunden) wird Do, 5. M\"arz, 10 Uhr an der Spiegelgasse festgelegt.\\
Vorl\"aufig gelten folgende Zust\"andigkeiten:
\begin{itemize}
\item Anna bereitet die PowerPoint für den Meilenstein 1 vor, und den Gantt-Projektplan.
\item Meipei macht das Mock-up unseres Spiels.
\item Johannes wird während der ganzen Projektdauer das Tagebuch betreuen.
\end{itemize}

\paragraph{Eintrag 3:}
28.02.2020\\
Meipei fügt unserem Repository ein gitignore-File hinzu und löscht die überflüssigen Dateien.\\
Adrian wird einen ersten Entwurf des Netzwerkprotokolls anfertigen und ins Repository stellen.

\paragraph{Eintrag 4:}
05.03.2020\\
Treffen der gesamten Gruppe, um den Meilenstein 1 vorzubereiten. Wir arbeiten intensiv an der Powerpoint-Präsentation. Dabei ist zum Vorschein gekommen, dass wir den Spielablauf und die Regeln präzisieren müssen. Dazu ist es auch nötig, uns Gedanken zu machen über das Netzwerkprotokoll, und was für Züge und Befehle möglich sein sollen.\\
Meipei wird weiter am Mockup arbeiten, und Johannes wird das gitignore-File anpassen.\\
Die Arbeitsaufteilung wird folgendermassen präzisiert:
\begin{itemize}
\item Server: Adrian und Meipei
\item Client: Johannes und Anna
\item Tagebuch: Johannes
\item Präsentationen: Anna
\item GUI: Meipei\\
\end{itemize}

Nächste Treffen:
\begin{itemize}
\item Fr, 06. März, 14.00 Uhr im Gruppenarbeitsraum Spiegelgasse:\\
Präsentation weitermachen, Netzwerkprotokoll besprechen
\item Mo, 09. März, 08.30 Uhr im Gruppenarbeitsraum Spiegelgasse:\\
Netzwerkprotokoll, Code produzieren für Meilenstein 2
\item Mi, 11. März, 08.30 Uhr im Gruppenarbeitsraum Spiegelgasse:\\
Präsentation einüben
\end{itemize}

\end{document}