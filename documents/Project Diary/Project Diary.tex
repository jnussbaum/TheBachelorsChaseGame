\documentclass[12pt]{article}
\usepackage{color}
\usepackage{hyperref}
\pagestyle{plain}

\title{Programmierprojekt FS20\\Bachelor Chasers\\Diary}
\author{\\Adrian Prokopczyk\\Johannes Nussbaum\\Anna Diack\\Meipei Nghiem}
\date{}

\begin{document}
\begin{titlepage}

\maketitle
\thispagestyle{empty}
\setcounter{tocdepth}{2}

\end{titlepage}

\paragraph{Eintrag 1:}
20.02.2020 \\
Erster Termin der Einf\"uhrungsvorlesung. Sehr rasch nachdem wir unsere Gruppe gebildet hatten, einigten wir uns auf Black Jack als Basis für unser Spiel. Um es etwas spannender zu gestalten, m\"ochten wir Begriffe aus dem Studentenleben einbauen: Das Ziel ist es, 180 Punkte zu sammeln ("Bachelor"). Die Karten k\"onnten benannt werden nach Ereignissen im Studentenleben: Kaffee und Lernen gibt Pluspunkte; Partymachen und Schw\"anzen gibt Minuspunkte.\\
Wir erstellen eine WhatsApp-Gruppe zur Kommunikation untereinander.

\paragraph{Eintrag 2:}
26.02.2020\\
Vor der ersten \"Ubungsstunde trafen wir uns, um uns Gedanken zu machen \"uber unser Spiel. Wir beschliessen einige Veränderungen und Verfeinerungen, die Anna in einem Word-Dokument festh\"alt. Der Name unseres Spiels wird "Bachelor's Chase" sein, und wir sind die "Chaser".\\
Als n\"achstes Treffen (neben den Vorlesungsstunden) wird Do, 5. M\"arz, 10 Uhr an der Spiegelgasse festgelegt.\\
Vorl\"aufig gelten folgende Zust\"andigkeiten:
\begin{itemize}
\item Anna bereitet die PowerPoint für den Meilenstein 1 vor, und den Gantt-Projektplan.
\item Meipei macht das Mock-up unseres Spiels.
\item Johannes wird während der ganzen Projektdauer das Tagebuch betreuen.
\item Adrian f\"angt mit dem Networking an.
\end{itemize}

\paragraph{Eintrag 3:}
28.02.2020\\
Meipei f\"ugt unserem Repository ein gitignore-File hinzu und l\"oscht die überflüssigen Dateien.\\
Adrian wird einen ersten Entwurf des Netzwerkprotokolls anfertigen und ins Repository stellen.

\paragraph{Eintrag 4:}
05.03.2020\\
Treffen der gesamten Gruppe, um den Meilenstein 1 vorzubereiten. Wir arbeiten intensiv an der Powerpoint-Pr\"asentation. Dabei ist zum Vorschein gekommen, dass wir den Spielablauf und die Regeln pr\"azisieren m\"ussen. Dazu ist es auch n\"otig, uns Gedanken zu machen \"uber das Netzwerkprotokoll, und was f\"ur Z\"uge und Befehle m\"oglich sein sollen.\\
Meipei wird weiter am Mockup arbeiten, und Johannes wird das gitignore-File anpassen.\\
Die Arbeitsaufteilung wird folgendermassen pr\"azisiert:
\begin{itemize}
\item Server: Adrian und Meipei
\item Client: Johannes und Anna
\item Tagebuch: Johannes
\item Pr\"asentationen: Anna
\item GUI: Meipei und Anna\\
\end{itemize}

\noindent N\"achstes Treffen:
\begin{itemize}
\item Fr, 06. M\"arz, 14.00 Uhr im Gruppenarbeitsraum Spiegelgasse:\\
Pr\"asentation weitermachen, Netzwerkprotokoll besprechen
\item Mo, 09. M\"arz, 08.30 Uhr im Gruppenarbeitsraum Spiegelgasse:\\
Netzwerkprotokoll, Code produzieren f\"ur Meilenstein 2
\item Mi, 11. M\"arz, 08.30 Uhr im Gruppenarbeitsraum Spiegelgasse:\\
Pr\"asentation ein\"uben
\end{itemize}

\paragraph{Eintrag 5:}
06.03.2020\\
Treffen von Meipei, Adrian und Anna. Johannes ist krank und steht deshalb nur per WhatsApp-Chat zur Verf\"ugung.\\
Wir legen folgende Aufteilung für die Pr\"asentation fest:
\begin{itemize}
\item Anna: Spielidee, Spielbeschreibung/Regeln
\item Mei: Karten, Mockup
\item Adrian: Anforderungen, Client/Server
\item Johannes: Organisation/Software, Fragen/Abschluss
\end{itemize}

\noindent Die Pr\"asentation  ist jetzt fertig.\\
Mei hat das gitignore mithilfe des Mails des Tutors angepasst.\\
Adrian hat angefangen, Server und Client zu implementieren.\\ 
Anna hat das gantt fertiggestellt.

\paragraph{Eintrag 6:}
09.03.2020\\
Treffen von Meipei, Adrian und Anna. Johannes ist krank und schaltet sich per Skype zu.\\
Wir übten die Pr\"asentation ein und stoppten die Zeit. Für die kommenden Tage gilt die folgende Arbeitsteilung:
\begin{itemize} 
\item Mei und Adrian beginnen, den Chat zu programmieren.
\item Anna beginnt mit einem Client und versucht es an den Chat von Mei und Adrian anzupassen.
\item Johannes beginnt, das Ping-Pong zu programmieren.
\end{itemize}

\paragraph{Eintrag 6:}
11.03.2020\\
Treffen der gesamten Gruppe.\
Wir einigen uns darauf, Englisch zu verwenden für alle Variablen- und Methodennamen, Kommentare im Programm, und auch für die Anzeigetexte des Games. Das einzige, was Deutsch bleibt, sind die Commit messages, weil wir dort schon auf Deutsch begonnen haben.\

\end{document}