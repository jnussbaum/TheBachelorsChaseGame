\documentclass[12pt]{article}
\usepackage{color}
\usepackage{hyperref}
\pagestyle{plain}

\title{Programmierprojekt FS20\\The Bachelor's Chase\\Diary}
\author{\\Adrian Prokopczyk\\Johannes Nussbaum\\Anna Diack\\Meipei Nghiem}

\begin{document}
\[\[\maketitle
\setcounter{tocdepth}{2}

\end{titlepage}

\paragraph{Eintrag 1:}
Donnerstag, 20.02.2020 \\
Erster Termin der Einf\"uhrungsvorlesung. Sehr rasch nachdem wir unsere Gruppe gebildet hatten, einigten wir uns auf Black Jack als Basis für unser Spiel. Um es etwas spannender zu gestalten, m\"ochten wir Begriffe aus dem Studentenleben einbauen: Das Ziel ist es, 180 Punkte zu sammeln ("Bachelor"). Die Karten k\"onnten benannt werden nach Ereignissen im Studentenleben: Kaffee und Lernen gibt Pluspunkte; Partymachen und Schw\"anzen gibt Minuspunkte.\\
Wir erstellen eine WhatsApp-Gruppe zur Kommunikation untereinander.

\paragraph{Eintrag 2:}
Mittwoch, 26.02.2020\\
Vor der ersten \"Ubungsstunde trafen wir uns, um uns Gedanken zu machen \"uber unser Spiel. Wir beschliessen einige Veränderungen und Verfeinerungen, die Anna in einem Word-Dokument festh\"alt. Der Name unseres Spiels wird "Bachelor's Chase" sein, und wir sind die "Chaser".\\
Als n\"achstes Treffen (neben den Vorlesungsstunden) wird Do, 5. M\"arz, 10 Uhr an der Spiegelgasse festgelegt.\\
Vorl\"aufig gelten folgende Zust\"andigkeiten:
\begin{itemize}
\item Anna bereitet die PowerPoint für den Meilenstein 1 vor, und den Gantt-Projektplan.
\item Meipei macht das Mock-up unseres Spiels.
\item Johannes wird während der ganzen Projektdauer das Tagebuch betreuen.
\item Adrian f\"angt mit dem Networking an.
\end{itemize}

\paragraph{Eintrag 3:}
Freitag, 28.02.2020\\
Meipei f\"ugt unserem Repository ein gitignore-File hinzu und l\"oscht die \"uberfl\"ussigen Dateien.\\
Adrian wird einen ersten Entwurf des Netzwerkprotokolls anfertigen und ins Repository stellen.

\paragraph{Eintrag 4:}
Donnerstag, 05.03.2020\\
Treffen der gesamten Gruppe, um den Meilenstein 1 vorzubereiten. Wir arbeiten intensiv an der Powerpoint-Pr\"asentation. Dabei ist zum Vorschein gekommen, dass wir den Spielablauf und die Regeln pr\"azisieren m\"ussen. Dazu ist es auch n\"otig, uns Gedanken zu machen \"uber das Netzwerkprotokoll, und was f\"ur Z\"uge und Befehle m\"oglich sein sollen.\\
Meipei wird weiter am Mockup arbeiten, und Johannes wird das gitignore-File anpassen.\\
Die Arbeitsaufteilung wird folgendermassen pr\"azisiert:
\begin{itemize}
\item Server: Adrian und Meipei
\item Client: Johannes und Anna
\item Tagebuch: Johannes
\item Pr\"asentationen: Anna
\item tbc.GUI: Meipei und Anna\\
\end{itemize}

\noindent N\"achstes Treffen:
\begin{itemize}
\item Fr, 06. M\"arz, 14.00 Uhr im Gruppenarbeitsraum Spiegelgasse:\\
Pr\"asentation weitermachen, Netzwerkprotokoll besprechen
\item Mo, 09. M\"arz, 08.30 Uhr im Gruppenarbeitsraum Spiegelgasse:\\
Netzwerkprotokoll, Code produzieren f\"ur Meilenstein 2
\item Mi, 11. M\"arz, 08.30 Uhr im Gruppenarbeitsraum Spiegelgasse:\\
Pr\"asentation ein\"uben
\end{itemize}

\paragraph{Eintrag 5:}
Freitag, 06.03.2020\\
Treffen von Meipei, Adrian und Anna. Johannes ist krank und steht deshalb nur per WhatsApp-Chat zur Verf\"ugung.\\
Wir legen folgende Aufteilung für die Pr\"asentation fest:
\begin{itemize}
\item Anna: Spielidee, Spielbeschreibung/Regeln
\item Mei: Karten, Mockup
\item Adrian: Anforderungen, Client/Server
\item Johannes: Organisation/Software, Fragen/Abschluss
\end{itemize}

\noindent Die Pr\"asentation  ist jetzt fertig.\\
Mei hat das gitignore mithilfe des Mails des Tutors angepasst.\\
Adrian hat angefangen, Server und Client zu implementieren.\\ 
Anna hat das gantt fertiggestellt.

\paragraph{Eintrag 6:}
Montag, 09.03.2020\\
Treffen von Meipei, Adrian und Anna. Johannes ist krank und schaltet sich per Skype zu.\\
Wir übten die Pr\"asentation ein und stoppten die Zeit. Für die kommenden Tage gilt die folgende Arbeitsteilung:
\begin{itemize} 
\item Mei und Adrian beginnen, den Chat zu programmieren.
\item Anna beginnt mit einem Client und versucht es an den Chat von Mei und Adrian anzupassen.
\item Johannes beginnt, das Ping-Pong zu programmieren.
\end{itemize}

\paragraph{Eintrag 7:}
Mittwoch, 11.03.2020\\
Treffen der gesamten Gruppe.\\
Wir einigen uns darauf, Englisch zu verwenden f\"ur alle Variablen- und Methodennamen, Kommentare im Programm, und auch f\"ur die Anzeigetexte des Games. Das einzige, was Deutsch bleibt, sind die Commit messages, weil wir dort schon auf Deutsch begonnen haben.\\

\paragraph{Eintrag 8:}
Donnerstag, 12.03.2020\\
Wir treffen uns in einem Sitzungsraum, um den Vortrag aufzunehmen. Es klappt alles ausgezeichnet, sodass wir nach kurzer Zeit bereits die fertige Video-Datei haben mit den Teilen von allen Gruppenmitgliedern.

\paragraph{Eintrag 9:}
Montag, 16.03.2020\\
Telefonkonferenz auf Discord.\\
Der Handler hat noch kleinere Probleme, Adrian arbeitet weiter daran. Die anderen Klassen m\"ussen irgendwann noch angepasst werden, dass sie den Handler benutzen, statt direkt miteinander zu kommunizieren.\\
Der Chat funktioniert schon recht gut, ausser dass Umlaute noch  nicht funktionieren.\\
Das Netzwerkprotokoll steht gr\"osstenteils, und wird in den kommenden Tagen von Adrian in seinem Handler implementiert werden.\\
Mei k\"ummert sich um die Usernamen.\\
Anna k\"ummert sich um Hamachi.\\
Johannes verschiebt das Login und Logout vom Chat zum Handler.\\
Wir setzen uns Freitag, 20. M\"arz als Deadline für alle Anforderungen von Milestone 2. Sollte dann etwas nicht klappen, haben wir noch das Wochenende als Puffer.

\paragraph{Eintrag 10:}
Mittwoch, 18.03.2020\\
\"Ubungsstunde per Zoom. Lange Diskussion \"uber den Handler. Leider ist es uns noch unklar, wie wir die Kommunikation der Clients mit dem Server organisieren. Wir verbringen einen Grossteil des Nachmittags beim gemeinsamen Programmieren, d.h. Johannes teilt seinen Bildschirm und schreibt Code, und alle vier diskutieren dar\"uber, wie der Code geschrieben/abge\"andert werden soll.\\
Schliesslich vertagen wir das weitere Vorgehen auf morgen 13 Uhr (Ausweichtermin, weil Johannes eine Terminkollision hat: 16 Uhr)

\paragraph{Eintrag 11:}
Donnerstag, 19.03.2020\\
Wir beginnen um 13 Uhr ein Zoom-Meeting. Mei hat auf\\ https://github.com/RuthRainbow/Chat-Server/ ein gutes Beispiel gefunden f\"ur das, was wir brauchen. Nun versuchen wir unseren vorhandenen Code diesem Beispiel gem\"ass anzupassen.\\
Leider m\"ussen wir unsere Versuche aufgeben, unseren bisherigen Code zu verbessern. Wir beginnen bei Null, und einigen uns auf folgendes Modell:\\
Sowohl Server als auch Client werden als statische Klassen von der Konsole aus gestartet; sie haben ihren Code jeweils in ihrer main-Methode.\\
Der Client startet aus der main-Methode heraus ein ClientHandler-Objekt, das die gesamte Netzwerkkommunikation \"ubernimmt. Ebenfalls startet der Client einen ChatClient, welcher f\"ur den Chat verantwortlich ist.\\
Sobald der Server die einkommende Verbindung erhalten hat, erstellt er ein ServerHandler-Objekt, das fortan für die Kommunikation zwischen dem Server und *diesem* Client zust\"andig ist. Ebenso startet er einen Chatserver.\\
Die beiden Handler-Klassen haben zwei Aufgaben: Erstens lauschen sie in ihrer run()-Methode auf eingehende Nachrichten, entschl\"usseln diese ankommenden Strings gem\"ass Netzwerkprotokoll, und rufen die entsprechenden Methoden auf, um die Informationen weiterzuleiten. Zweitens haben sie Methoden, die vom dahinterstehenden Server/Client aufgerufen werden k\"onnen, um eine Nachricht gem\"ass Netzwerkprotokoll rauszuschicken.\\
Wir verwenden den ganzen Nachmittag bis 18 Uhr, um im Pair-Programming diese ganze Architektur umzusetzen. Johannes tippt und gibt seinen Bildschirm frei, während die anderen ihm laufend Feedback und Korrekturen mitteilen.\\
Das n\"achste Zoom-Meeting ist morgen nach der Vorlesung.

\paragraph{Eintrag 12:}
Freitag, 20.03.2020\\
Nach der Vorlesung starten wir ein Zoom-Meeting. Wiederum arbeiten wir im Pair-Programming, wobei Johannes tippt, und die anderen Feedback geben. Bis um ca. 14 Uhr haben wir die meisten Fehler beseitigt, sodass s\"amtliche Klassen fehlerfrei kompilieren und einigermassen benutzbar sind.\\
Morgen um 10 Uhr wollen wir fortfahren, um bestehende Schw\"achen zu beseitigen, und unser Programm umfassend zu testen.

\paragraph{Eintrag 13:}
Samstag, 21.03.2020\\
Wir beginnen um 10 Uhr ein Zoom-Meeting. Zun\"achst machen wir weiter im Pair-Programming (Johannes tippt). Als die Grundstruktur fehlerfrei steht und keine gr\"osseren Bugs mehr vorhanden sind, beschliessen wir, für die \"ubrigbleibende Detailarbeit in Einzelarbeit fortzufahren. Auf GitLab erstellen wir Issues, um den \"Uberblick zu haben, was bis Meilenstein 2 erledigt sein muss: 
\begin{itemize} 
\item Johannes aktualisiert das Tagebuch.
\item Anna implementiert das Logout.
\item Meipei verfeinert den Chat (Localhostabfrage, private Message etc.) und implementiert ebenfalls am Logout
\item Adrian stellt sicher, dass die Protokollbefehle validiert werden vor ihrer Ausf\"uhrung.
\item Die \"ubrigen Issues auf GitLab sind noch offen, und jede(r) kann sich eines zuweisen, um daran zu arbeiten.
\end{itemize}

\paragraph{Eintrag 14:}
Sonntag, 22.03.2020\\
Meipei und Anna beginnen um 14 Uhr ein Discord-Meeting. Zu Beginn rekapitulieren wir nochmals die Grundprinzipien der Threads und Streams innerhalb des Codes um via Pair-Programming das Logout für den Chat sinnvoll implementieren zu können. Des Weiteren schauen wir uns die Achievements für den Meilenstein II an, um abzugleichen was wir schon erledigt haben und was noch offen steht. Diesbezüglich gibt es noch Issues auf GitLab mit internen Erledigungen die wir für uns noch erledigen wollten.
Heute haben wir:
\begin{itemize}
\item Logout implementiert
\item 2. Hashmap aus dem ChatServer rausgenommen
\item Diary-Eintrag ergänzt
\item Netzwerkprotokoll angepasst
\item Richtiges \"Uberpr\"ufen der Localhostabfrage
\end{itemize}\]

\paragraph{Eintrag 15:}
Mittwoch, 25.03.20\\
Um 14 Uhr loggen sich alle für die \"Ubungsstunde in Zoom ein, um Meilenstein II vorzuzeigen. Nachdem wir alles vorgezeigt haben, diskutieren wir welche Achievements bis zum dritten Meilenstein zu erreichen sind. Dabei kommen fragen auf, weil die Formulierung mancher Achievements für uns nicht eindeutig sind. Deshalb fragen wir unseren Tutor diese genauer zu erläutern. Simon erkl\"art sie ganz gut und wir f\"uhren mit unserer Aufteilung fort. Für den kommenden Freitag werden sich Adrian und Johannes um die Implemntierung der Spiellogic kümmern, während Meipei und Anna die GUI für den Chat schonmal beginnen. 

\paragraph{Eintrag 16:}
Freitag, 27.03.20\\
Um 9 Uhr fangen wir getrennt mit unserer Arbeit an.\\
Meipei und Anna beginnen mit ihrem Discord-Meeting und implementieren die Anfangs-GUI für den Chat. 
Das Fenster öffnet sich und die Auswahl, wie wir sie im Mockup bei Präsentation I gezeigt haben, Start, Ziel, Karten, Regeln und Einstellungen ist zu sehen.
Nun schauen wir, dass sich weitere Fenster öffnen sobald man auf eine Auswahl klickt.\]

\paragraph{Eintrag 17:}
Montag, 30.03.20\\
Um 13 Uhr starten wir ein Zoom-Meeting indem wir alle uns über den gegenwertigen Zustand austauschen. Johannes und Adrian sind für die Spiellogik zuständig und fragen die GUI-Gruppe wie weit sie gekommen sind und welche Informationen sie für das GUI brauchen. Die GUI Gruppe schlägt der Logik Gruppe vor die Hashmaps zu reduzieren damit nicht so viele Fehler auftauchen und die Lesbarkeit besser wird. 

\paragraph{Eintrag 18:}
Mittwoch, 01.04.20\\
Um 10.30 Uhr beginnen Mei und Anna ihr Zoom-Meeting um weiter an der GUI zu arbeiten. Adrian und Johannes beginnen ebenfalls separat mit ihrem Meeting.
Um 14 Uhr beginnt das Zoom-Meeting für die Übungsstunde. Dort tauschen sich beide Gruppen nochmal aus wie weit sie gekommen sind und welche Fragen noch offen sind. Unser Tutor Simon stellt nochmal für alle klar, welche Vorraussetzungen bis kommenden Montag erfüllt sein müssen.
\]
\end{document}