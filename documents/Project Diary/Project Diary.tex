\documentclass[12pt]{article}
\usepackage{color}
\usepackage{hyperref}
\pagestyle{plain}

\title{Programmierprojekt FS20\\The Bachelor's Chase\\Diary}
\author{\\Adrian Prokopczyk\\Johannes Nussbaum\\Anna Diack\\Meipei Nghiem}

\begin{document}
\begin{titlepage}
\maketitle
\end{titlepage}

\setcounter{tocdepth}{2}


\paragraph{Eintrag 1:}
Donnerstag, 20.02.2020 \\
Erster Termin der Einf\"uhrungsvorlesung. Sehr rasch nachdem wir unsere Gruppe gebildet hatten, einigten wir uns auf Black Jack als Basis f\"ur unser Spiel. Um es etwas spannender zu gestalten, m\"ochten wir Begriffe aus dem Studentenleben einbauen: Das Ziel ist es, 180 Punkte zu sammeln ("Bachelor"). Die Karten k\"onnten benannt werden nach Ereignissen im Studentenleben: Kaffee und Lernen gibt Pluspunkte; Partymachen und Schw\"anzen gibt Minuspunkte.\\
Wir erstellen eine WhatsApp-Gruppe zur Kommunikation untereinander.

\paragraph{Eintrag 2:}
Mittwoch, 26.02.2020\\
Vor der ersten \"Ubungsstunde trafen wir uns, um uns Gedanken zu machen \"uber unser Spiel. Wir beschliessen einige Ver\"anderungen und Verfeinerungen, die Anna in einem Word-Dokument festh\"alt. Der Name unseres Spiels wird "Bachelor's Chase" sein, und wir sind die "Chaser".\\
Als n\"achstes Treffen (neben den Vorlesungsstunden) wird Do, 5. M\"arz, 10 Uhr an der Spiegelgasse festgelegt.\\
Vorl\"aufig gelten folgende Zust\"andigkeiten:
\begin{itemize}
\item Anna bereitet die PowerPoint f\"ur den Meilenstein 1 vor, und den Gantt-Projektplan.
\item Meipei macht das Mock-up unseres Spiels.
\item Johannes wird w\"ahrend der ganzen Projektdauer das Tagebuch betreuen.
\item Adrian f\"angt mit dem Networking an.
\end{itemize}

\paragraph{Eintrag 3:}
Freitag, 28.02.2020\\
Meipei f\"ugt unserem Repository ein gitignore-File hinzu und l\"oscht die \"uberfl\"ussigen Dateien.\\
Adrian wird einen ersten Entwurf des Netzwerkprotokolls anfertigen und ins Repository stellen.

\paragraph{Eintrag 4:}
Donnerstag, 05.03.2020\\
Treffen der gesamten Gruppe, um den Meilenstein 1 vorzubereiten. Wir arbeiten intensiv an der Powerpoint-Pr\"asentation. Dabei ist zum Vorschein gekommen, dass wir den Spielablauf und die Regeln pr\"azisieren m\"ussen. Dazu ist es auch n\"otig, uns Gedanken zu machen \"uber das Netzwerkprotokoll, und was f\"ur Z\"uge und Befehle m\"oglich sein sollen.\\
Meipei wird weiter am Mockup arbeiten, und Johannes wird das gitignore-File anpassen.\\
Die Arbeitsaufteilung wird folgendermassen pr\"azisiert:
\begin{itemize}
\item Server: Adrian und Meipei
\item Client: Johannes und Anna
\item Tagebuch: Johannes
\item Pr\"asentationen: Anna
\item tbc.gui: Meipei und Anna\\
\end{itemize}

\noindent N\"achstes Treffen:
\begin{itemize}
\item Fr, 06. M\"arz, 14.00 Uhr im Gruppenarbeitsraum Spiegelgasse:\\
Pr\"asentation weitermachen, Netzwerkprotokoll besprechen
\item Mo, 09. M\"arz, 08.30 Uhr im Gruppenarbeitsraum Spiegelgasse:\\
Netzwerkprotokoll, Code produzieren f\"ur Meilenstein 2
\item Mi, 11. M\"arz, 08.30 Uhr im Gruppenarbeitsraum Spiegelgasse:\\
Pr\"asentation ein\"uben
\end{itemize}

\paragraph{Eintrag 5:}
Freitag, 06.03.2020\\
Treffen von Meipei, Adrian und Anna. Johannes ist krank und steht deshalb nur per WhatsApp-Chat zur Verf\"ugung.\\
Wir legen folgende Aufteilung f\"ur die Pr\"asentation fest:
\begin{itemize}
\item Anna: Spielidee, Spielbeschreibung/Regeln
\item Mei: Karten, Mockup
\item Adrian: Anforderungen, Client/Server
\item Johannes: Organisation/Software, Fragen/Abschluss
\end{itemize}

\noindent Die Pr\"asentation  ist jetzt fertig.\\
Mei hat das gitignore mithilfe des Mails des Tutors angepasst.\\
Adrian hat angefangen, Server und Client zu implementieren.\\ 
Anna hat das gantt fertiggestellt.

\paragraph{Eintrag 6:}
Montag, 09.03.2020\\
Treffen von Meipei, Adrian und Anna. Johannes ist krank und schaltet sich per Skype zu.\\
Wir \"ubten die Pr\"asentation ein und stoppten die Zeit. F\"ur die kommenden Tage gilt die folgende Arbeitsteilung:
\begin{itemize} 
\item Mei und Adrian beginnen, den Chat zu programmieren.
\item Anna beginnt mit einem Client und versucht es an den Chat von Mei und Adrian anzupassen.
\item Johannes beginnt, das Ping-Pong zu programmieren.
\end{itemize}

\paragraph{Eintrag 7:}
Mittwoch, 11.03.2020\\
Treffen der gesamten Gruppe.\\
Wir einigen uns darauf, Englisch zu verwenden f\"ur alle Variablen- und Methodennamen, Kommentare im Programm, und auch f\"ur die Anzeigetexte des Games. Das einzige, was Deutsch bleibt, sind die Commit messages, weil wir dort schon auf Deutsch begonnen haben.\\

\paragraph{Eintrag 8:}
Donnerstag, 12.03.2020\\
Wir treffen uns in einem Sitzungsraum, um den Vortrag aufzunehmen. Es klappt alles ausgezeichnet, sodass wir nach kurzer Zeit bereits die fertige Video-Datei haben mit den Teilen von allen Gruppenmitgliedern.

\paragraph{Eintrag 9:}
Montag, 16.03.2020\\
Telefonkonferenz auf Discord.\\
Der Handler hat noch kleinere Probleme, Adrian arbeitet weiter daran. Die anderen Klassen m\"ussen irgendwann noch angepasst werden, dass sie den Handler benutzen, statt direkt miteinander zu kommunizieren.\\
Der Chat funktioniert schon recht gut, ausser dass Umlaute noch  nicht funktionieren.\\
Das Netzwerkprotokoll steht gr\"osstenteils, und wird in den kommenden Tagen von Adrian in seinem Handler implementiert werden.\\
Mei k\"ummert sich um die Usernamen.\\
Anna k\"ummert sich um Hamachi.\\
Johannes verschiebt das Login und Logout vom Chat zum Handler.\\
Wir setzen uns Freitag, 20. M\"arz als Deadline f\"ur alle Anforderungen von Milestone 2. Sollte dann etwas nicht klappen, haben wir noch das Wochenende als Puffer.

\paragraph{Eintrag 10:}
Mittwoch, 18.03.2020\\
\"Ubungsstunde per Zoom. Lange Diskussion \"uber den Handler. Leider ist es uns noch unklar, wie wir die Kommunikation der Clients mit dem Server organisieren. Wir verbringen einen Grossteil des Nachmittags beim gemeinsamen Programmieren, d.h. Johannes teilt seinen Bildschirm und schreibt Code, und alle vier diskutieren dar\"uber, wie der Code geschrieben/abge\"andert werden soll.\\
Schliesslich vertagen wir das weitere Vorgehen auf morgen 13 Uhr (Ausweichtermin, weil Johannes eine Terminkollision hat: 16 Uhr)

\paragraph{Eintrag 11:}
Donnerstag, 19.03.2020\\
Wir beginnen um 13 Uhr ein Zoom-Meeting. Mei hat auf\\ https://github.com/RuthRainbow/Chat-Server/ ein gutes Beispiel gefunden f\"ur das, was wir brauchen. Nun versuchen wir unseren vorhandenen Code diesem Beispiel gem\"ass anzupassen.\\
Leider m\"ussen wir unsere Versuche aufgeben, unseren bisherigen Code zu verbessern. Wir beginnen bei Null, und einigen uns auf folgendes Modell:\\
Sowohl Server als auch Client werden als statische Klassen von der Konsole aus gestartet; sie haben ihren Code jeweils in ihrer main-Methode.\\
Der Client startet aus der main-Methode heraus ein ClientHandler-Objekt, das die gesamte Netzwerkkommunikation \"ubernimmt. Ebenfalls startet der Client einen ChatClient, welcher f\"ur den Chat verantwortlich ist.\\
Sobald der Server die einkommende Verbindung erhalten hat, erstellt er ein ServerHandler-Objekt, das fortan f\"ur die Kommunikation zwischen dem Server und *diesem* Client zust\"andig ist. Ebenso startet er einen Chatserver.\\
Die beiden Handler-Klassen haben zwei Aufgaben: Erstens lauschen sie in ihrer run()-Methode auf eingehende Nachrichten, entschl\"usseln diese ankommenden Strings gem\"ass Netzwerkprotokoll, und rufen die entsprechenden Methoden auf, um die Informationen weiterzuleiten. Zweitens haben sie Methoden, die vom dahinterstehenden Server/Client aufgerufen werden k\"onnen, um eine Nachricht gem\"ass Netzwerkprotokoll rauszuschicken.\\
Wir verwenden den ganzen Nachmittag bis 18 Uhr, um im Pair-Programming diese ganze Architektur umzusetzen. Johannes tippt und gibt seinen Bildschirm frei, w\"ahrend die anderen ihm laufend Feedback und Korrekturen mitteilen.\\
Das n\"achste Zoom-Meeting ist morgen nach der Vorlesung.

\paragraph{Eintrag 12:}
Freitag, 20.03.2020\\
Nach der Vorlesung starten wir ein Zoom-Meeting. Wiederum arbeiten wir im Pair-Programming, wobei Johannes tippt, und die anderen Feedback geben. Bis um ca. 14 Uhr haben wir die meisten Fehler beseitigt, sodass s\"amtliche Klassen fehlerfrei kompilieren und einigermassen benutzbar sind.\\
Morgen um 10 Uhr wollen wir fortfahren, um bestehende Schw\"achen zu beseitigen, und unser Programm umfassend zu testen.

\paragraph{Eintrag 13:}
Samstag, 21.03.2020\\
Wir beginnen um 10 Uhr ein Zoom-Meeting. Zun\"achst machen wir weiter im Pair-Programming (Johannes tippt). Als die Grundstruktur fehlerfrei steht und keine gr\"osseren Bugs mehr vorhanden sind, beschliessen wir, f\"ur die \"ubrigbleibende Detailarbeit in Einzelarbeit fortzufahren. Auf GitLab erstellen wir Issues, um den \"Uberblick zu haben, was bis Meilenstein 2 erledigt sein muss: 
\begin{itemize} 
\item Johannes aktualisiert das Tagebuch.
\item Anna implementiert das Logout.
\item Meipei verfeinert den Chat (Localhostabfrage, private Message etc.) und implementiert ebenfalls am Logout
\item Adrian stellt sicher, dass die Protokollbefehle validiert werden vor ihrer Ausf\"uhrung.
\item Die \"ubrigen Issues auf GitLab sind noch offen, und jede(r) kann sich eines zuweisen, um daran zu arbeiten.
\end{itemize}

\paragraph{Eintrag 14:}
Sonntag, 22.03.2020\\
Meipei und Anna beginnen um 14 Uhr ein Discord-Meeting. Zu Beginn rekapitulieren wir nochmals die Grundprinzipien der Threads und Streams innerhalb des Codes, um im Pair-Programming das Logout f\"ur den Chat sinnvoll implementieren zu k\"onnen. Des Weiteren schauen wir uns die Achievements f\"ur den Meilenstein II an, um abzugleichen, was wir schon erledigt haben und was noch offen steht. Diesbez\"uglich gibt es noch Issues auf GitLab mit Dingen, die zu erledigen sind.
Heute haben wir:
\begin{itemize}
\item Logout implementiert
\item 2. Hashmap aus dem ChatServer rausgenommen
\item Diary-Eintrag erg\"anzt
\item Netzwerkprotokoll angepasst
\item Richtiges \"Uberpr\"ufen der Localhostabfrage
\end{itemize}

\paragraph{Eintrag 15:}
Mittwoch, 25.03.20\\
Um 14 Uhr loggen sich alle f\"ur die \"Ubungsstunde in Zoom ein, um Meilenstein II vorzuzeigen. Nachdem wir alles vorgezeigt haben, diskutieren wir, welche Achievements bis zum dritten Meilenstein zu erreichen sind. Dabei kommen Fragen auf, weil die Formulierung mancher Achievements f\"ur uns nicht eindeutig sind. Deshalb bitten wir unseren Tutor, diese genauer zu erl\"autern. Simon erkl\"art sie ganz gut und wir fahren mit unserer Aufteilung fort. F\"ur den kommenden Freitag werden sich Adrian und Johannes um die Implementierung der Spiellogik k\"ummern, w\"ahrend Meipei und Anna die GUI f\"ur den Chat in Angriff nehmen. 

\paragraph{Eintrag 16:}
Freitag, 27.03.20\\
Um 9 Uhr fangen wir getrennt mit unserer Arbeit an.\\
Meipei und Anna beginnen mit ihrem Discord-Meeting und implementieren die Anfangs-GUI f\"ur den Chat. Das Fenster \"offnet sich und die Auswahl erscheint, wie wir sie im Mockup bei Pr\"asentation I gezeigt haben: Start, Ziel, Karten, Regeln und Einstellungen.Nun schauen wir, dass sich weitere Fenster \"offnen, sobald man auf eine Auswahl klickt.\\
Johannes und Adrian beginnen mit der Spiellogik. Dazu \"uberlegen wir uns die Abl\"aufe, und welche Klassen daf\"ur n\"otig sind. Am Ende des Tages haben wir aber noch nichts, was funktionsf\"ahig w\"are.

\paragraph{Eintrag 17:}
Montag, 30.03.20\\
Um 13 Uhr starten wir ein Zoom-Meeting, in welchem wir alle uns \"uber den gegenw\"artigen Stand der Dinge austauschen. Johannes und Adrian sind f\"ur die Spiellogik zust\"andig und fragen die GUI-Gruppe, wie weit sie gekommen sind, und welche Informationen sie f\"ur das GUI brauchen. Die GUI-Gruppe schl\"agt der Logik-Gruppe vor, f\"ur die Spielerverwaltung eine eigene Player-Klasse zu implementieren statt der vielen verschiedenen Hashmaps, damit nicht so viele Fehler auftauchen und die Lesbarkeit besser wird.\\
Adrian und Johannes setzen diesen Vorschlag anschliessend um und arbeiten weiter an der Spiellogik.

\paragraph{Eintrag 18:}
Mittwoch, 01.04.20\\
Um 10.30 Uhr beginnen Mei und Anna ihr Zoom-Meeting, um weiter an der GUI zu arbeiten. Adrian und Johannes beginnen ebenfalls separat mit ihrem Meeting.
Um 14 Uhr beginnt das Zoom-Meeting f\"ur die \"Ubungsstunde. Dort tauschen sich beide Gruppen nochmal aus, wie weit sie gekommen sind, und welche Fragen noch offen sind. Unser Tutor Simon stellt nochmal f\"ur alle klar, welche Vorraussetzungen bis kommenden Montag erf\"ullt sein m\"ussen.\\
Am Abend haben Johannes uns Adrian eine erste funktionierende Version des Spiels, die allerdings noch viele Fehler enth\"alt.

\paragraph{Eintrag 19:}
Donnerstag, 02.04.2020\\
Um 15 Uhr treffen sich Johannes und Adrian, um im Pair Programming weiterzufahren mit dem Spiel, was vor allem Debugging bedeutet. Bis 18 Uhr schaffen wir es, viele Fehler auszumerzen.\\
Um 18 Uhr starten Mei und Anna ihr Zoom-Meeting, um herauszufinden, weshalb sie den Usernamen in Ihrer GUI nicht ausgegeben kriegen. Des Weiteren suchen sie eine M\"oglichkeit, auf die FXML Dateien zuzugreifen, die ben\"otigt werden, um den Chat auf der GUI anzuzeigen. Den Fehler konnten sie finden und planen die n\"achsten Schritte f\"ur das n\"achste Zoom-Meeting.

\paragraph{Eintrag 20:}
Freitag, 03.04.2020\\
Um 13 Uhr treffen sich Meipei und Anna via Zoom-Meeting mit unserem Tutor Simon. Kurz danach schalten sich Johannes und Adrian auch dazu. Nachdem Meipei und Anna ihr Problem mit der Anzeige des Chats im GUI beheben konnten, unterh\"alt sich die ganze Gruppe \"uber die fehlenden Dokumente f\"ur Meilenstein III. Das n\"achste Zoom-Meeting als Gruppe setzen wir f\"ur den n\"achsten Tag.\\
Meipei und Anna bleiben weiter auf Zoom, um den Fehler des Usernames in der GUI anzugehen.\\
Johannes und Adrian entwickeln weiterhin die Spiellogik.

\paragraph{Eintrag 21:}
Samstag, 04.04.2020\\
Adrian und Johannes treffen sich um 15 Uhr mit dem Tutor, um Probleme zu besprechen. Anschliessend merzen sie die letzten Fehler aus der Spiellogik aus, sodass am Abend ein funktionsf\"ahiges Spiel zur Verf\"ugung steht.

\paragraph{Eintrag 22:}
Sonntag, 05.04.2020\\
Um 14 Uhr treffen wir uns alle, um zu besprechen, was noch alles getan werden muss bis morgen. Zusammen besprechen wir die Dokumente. Bis auf einige Kleinigkeiten sind sie in Ordnung.\\
Mei und Anna arbeiten daran, eine Liste von Lobbies und Spielern auszugeben, und den Broadcast zu implementieren.\\
Johannes und Adrian verwenden nochmals einige Zeit im Pair Programming, um das Spiel weiter zu debuggen, weil neue Fehler aufgetaucht sind.

\paragraph{Eintrag 23:}
Mittwoch, 08.04.2020\\
Um 14 Uhr beginnt unsere \"Ubungsstunde zur Bewertung von Meilenstein 3. In der Wartezeit besprechen wir die letzten Details und das weitere Vorgehen.

\paragraph{Eintrag 24:}
Donnerstag, 09.04.2020\\
Um 09 Uhr treffen sich Adrian und Johannes zum Pair Programming, um an der Spiellogik weiter zu programmieren. Wir setzen einige Kritikpunkte von der gestrigen Bewertung um, beispielsweise: Anstatt nur eine Runde auszusetzen, lautet die Option jetzt, aus dem aktuellen Match auszusteigen. Die Hauptherausforderung ist, dass sich der Client nicht beenden sollte, wenn der erste Match vor\"uber ist. Die Fehlersuche ist sehr hartn\"ackig. Wir bitten den Tutor per Mail um Hilfe.

\paragraph{Eintrag 25:}
Montag, 13.04.2020\\
Um 10 Uhr treffen sich Anna und Mei zu einem Zoom-Meeting und arbeiten zusammen an der GUI. Das Problem, dass die Lobby und die Spieler jeweils auf einer Liste angezeigt werden macht noch Probleme, welche sich nicht so leicht ausgeben lassen. Mit der Hilfe von Adrian, der f\"ur die Spiellogik zust\"andig ist, konnten wir die richtigen Listen und Variablennamen nutzen, um die beiden Listen ausgeben lassen zu k\"onnen. Allerdings fiel uns auf, dass es beim Updaten der aktuellen Liste noch Fehler ausgibt. Wir konnten uns an die platform-Methode erinnern, die unser Tutor Simon mal erw\"ahnt hatte, und konnten somit auch dieses Problem l\"osen. Das n\"achste Meeting als gesamte Gruppe wird am Mittwoch sein, um die Pr\"asentation fertig zustellen und \"uber die Demo zu sprechen.

\paragraph{Eintrag 26:}
Dienstag, 14.04.2020\\
Um 16 Uhr treffen sich Johannes und Adrian zum Pair Programming. Dank eines Inputs des Tutors schaffen wir es, den Fehler zu finden, sodass wir am Abend die Spiellogik fertig haben, inklusive mehrere Matches hintereinander.

\paragraph{Eintrag 27:}
Mittwoch, 15.04.2020\\
Erstmals seit einer Woche treffen wir uns wieder als gesamtes Team. Wir besprechen, wie wir die verschiedenen Branches ineinander mergen wollen, dass f\"ur die Pr\"asentation morgen alles stimmt. Wir mergen Branch Spiellogik in Master, und erstellen dann von master aus einen Demo-Branch.\\ Anna hat ein UML vorbereitet, f\"ur das wir aber noch einige \"Anderungen festlegen.\\
Entscheidungen:
\begin{itemize}
\item HighScore Liste wird von GUI Team implementiert, nicht von Spiellogik-Team.
\item F\"ur die Demo morgen haben wir in einem Drehbuch festgelegt, welche Karten im Deck vorkommen, sodass ein guter Demo-Effekt erzielt wird. 
\item Der Server wird via VPN zur Verf\"ugung gestellt, Johannes wird zwei Clients starten und in der Konsole vorzeigen.
\item Zus\"atzliches Metrics f\"ur QA: Lines of Code pro Methode
\end{itemize}

\paragraph{Eintrag 28:}
Donnerstag, 16.04.2020\\
Pr\"asentationstag! Leider wird die Pr\"asentation \"uberschattet von einem Verbindungsunterbruch im Zoom-Meeting, und dass die Demo nicht genau so funktioniert, wie wir das geplant h\"atten. Sonst l\"auft alles gut.

\paragraph{Eintrag 29:}
Dienstag, 21.04.2020\\
Treffen von Adrian, Meipei und Anna um 13.00 Uhr via Zoom. Wir rekapitulieren den gegenw\"artigen Stand des Projekts und vergegenw\"artigen die Achievements f\"ur Meilenstein IV. Uns f\"allt auf, dass wir noch das Spiel komplett auf der GUI spielbar machen m\"ussen, sowie einen Highscore, der sichtbar sein sollte. In der Spiellogik m\"ussen noch ein paar Regeln verfeinert werden und ein Timer, sowie die UnitTests m\"ussen noch implementiert werden. Wir setzen ein Gruppentermin f\"ur den kommenden Freitag.

\paragraph{Eintrag 30:}
Mittwoch, 22.04.2020\\
Mei und Anna treffen sich um 10.00 Uhr via Zoom, um eine korrekte Kartenanzeige auf der GUI zu implementieren. Es ist nicht einfach mit dem ImageView umzugehen und das Spielfeld sinnvoll zu gestalten. Wir schreiben uns die Fragen, die sich ergeben haben auf und fragen den Tutor in der heutigen \"Ubungsstunde.\\
Um 14.00 Uhr ist \"Ubungsstunde, wo wir Inputs vom Tutor erhalten. Johannes weiss noch nicht, wie er die Unit Tests sinnvoll aufbauen k\"onnte, und bekommt Hinweise vom Tutor, wie er es machen k\"onnte. Anschliessen arbeitet er separat an den JUnit Tests weiter.

\paragraph{Eintrag 31:}
Donnerstag, 23.04.2020\\
Mei und Anna treffen sich um 13.30 Uhr via Zoom und setzen sich an die Spiellogik. Bis 21.00 Uhr haben wir es geschafft, die Karten korrekt auf dem Spielfeld anzuzeigen und die restlichen Elemente \"uber die GUI anzeigen zu lassen. Das einzige was noch nicht richtig funktioniert ist der Highscore. Wir nehmen uns vor, am n\"achsten Tag die anderen der Gruppe zu fragen, welche Variablen den Inhalt der Clients wiedergeben, die sich innerhalb derselben Lobby befinden.

\paragraph{Eintrag 32:}
Freitag, 24.04.2020\\
Treffen zu viert. Wir fassen folgende Beschl\"usse:
\begin{itemize}
\item Adrian und Anna werden das Game vorzeigen in der \"Ubungsstunde, mit Hilfe von Hamachi.
\item Johannes und Adrian implementieren heute noch den Timer und den High Score. Anschliessend informieren sie Mei, damit sie diese Sachen ins GUI \"ubernehmen kann.
\item Johannes und Adrian setzen den Anfangsstand der Coins auf 0, stellen sicher, dass das Spiel spielbar ist f\"ur 2-4 Spieler und passen die Kartenh\"aufigkeit an (es hat zuviele Minuskarten mit zu vielen Minuspunkten). Ebenso passen sie die Textausgabe an, dass man die aktuellen Punkte angezeigt bekommt nach dem Ziehen einer Karte (und nicht vor dem Ziehen).
\item Wenn man weniger als 180 Punkte erreicht, bekommt man x-50 Coins, aber keine Minus-Coins.
\item Wir haben einen DROPPEDOUT Befehl vom Server an Client ins Netzwerkprotokoll \"ubernommen. Johannes wird das Dokument anpassen.
\item Es ist jetzt nicht mehr m\"oglich, einer Lobby beizutreten, wenn ein Spiel am Laufen ist. Das Netzwerkprotokoll wurde erg\"anzt um den Befehl REJECTTOJOINLOBBY.
\end{itemize}

\paragraph{Eintrag 33:}
Samstag, 25.04.2020\\
Wir treffen uns um 13.30 Uhr als ganze Gruppe. Gemeinsam knobeln wir an den letzten Schwierigkeiten im Game: Das Fenster, das den Spieler zu seinem Zug auffordert, erscheint zuerst zu oft, und dann zu selten. Nach langem Suchen finden wir die Fehler, sie waren beide in der Spiellogik.\\
In Einzelarbeit erledigen wir anschliessend kleinere Aufgaben: Johannes macht die Unittests fertig, Mei implementiert den Highscore mithilfe von Anregungen Adrians, Adrian f\"ugt die noch fehlenden Javadocs in in allen Nicht-GUI-Klassen hinzu, und Anna macht Messungen f\"ur die QA-Metrics.

\paragraph{Eintrag 34:}
Mittwoch, 29.04.2020\\
In der \"Ubungsstunde haben wir Meilenstein 4-Pr\"asentation.\\
Die Bilanz ist durchzogen, es gibt einige Kritikpunkte:
\begin{itemize}
\item Die Javadoc-Dokumentation muss noch ausf\"uhrlicher sein, sodass auch Exceptions und illegale Eingabwerte dokumentiert sind.
\item Unsere Unit-Tests sind unzul\"anglich, weil sie nur wenige Methoden testen, statt der gesamten Unit. Um dies zu \"andern, m\"ussten wir mit Mockito arbeiten und substanzielle \"Anderungen an der Architektur vornehmen, n\"amlich eine Entflechtung der Klassen, sodass die Komponenten in sich geschlossen funktionieren, und damit besser testbar sind. Alle relevanten Teile/Methoden/Vorg\"ange der Komponente m\"ussen getestet werden, nicht nur einige ausgew\"ahlte. Statt des ben\"otigten Sockets k\"onnte ein Mock-Objekt erstellt werden.
\item Der HighScore muss persistent sein, d.h. \"uber den Neustart des Servers hinaus verf\"ugbar sein.
\end{itemize}

\paragraph{Eintrag 35:}
Samstag, 02.05.2020\\
Anna, Meipei und Johannes treffen sich zu einer Lagebesprechung. Wir treffen folgende Beschl\"usse/Aufgabenverteilung:
\begin{itemize}
\item Unit-Tests auf andern Computern lauff\"ahig machen, aber sonst sein lassen: Johannes
\item Javadoc: auch Exceptions dokumentieren und komplexere Getter, und was passiert wenn man negative Werte eingibt statt den erwarteten positiven: Johannes
\item Archiving/Outreach folder: Anna
\item Cheatcode zum gewinnen: Adrian
\item Highscore persistent machen: Meipei
\item Manual + QA: Korrekturlesen kurz vor Abgabetermin: Johannes.
\item Lessons Learned in Powerpoint erg\"anzen: alle
\item In der Pr\"asentation die richtigen Prozentzahlen der QA-Funktionalit\"at eintragen: Johannes
\item F\"ur den Bonus-Punkt "Bug- und Issue-Tracker": Bitte benutzt alle die Issues in GitLab, und tragt eure Aufgaben dort ein, und schliesst die Issues, wenn sie erledigt sind. Auch die Bugs dort melden.
\item Free as in Speech 5 Bonus-Punkte: RedBull auf Energy Drink \"andern: Anna (GUI) + Adrian (Spiellogik)
\item Im Netzwerkprotokoll-Dokument ein Beispiel einf\"ugen: Adrian
\item Tutor fragen: All external libraries in your project are managed by gradle via maven central
\item Tutor fragen: Final version of network protocol is completely defined and documented in source code: Muss es Kommentare im Code haben?
\item Tutor fragen: Wie kann Socket f\"ur Tests erstellt werden?
\end{itemize}

\end{document}